%%%%%%%%%%%%%%%%%%%%%%%%%%%%%%%%%%%%%%%%%%%%%%%%%%%%%%%%%%%%%%
%% LaTeX template for the science justification to be       %%
%%       submitted as part of an ALMA proposal.             %%
%%                                                          %%
%%                      ALMA Cycle 2                        %%
%%                                                          %%
%%%%%%%%%%%%%%%%%%%%%%%%%%%%%%%%%%%%%%%%%%%%%%%%%%%%%%%%%%%%%%

%%%%%%%%%%%%%%%%%%%%%%%%%%%%%%%%%%%%%%%%%%%%%%%%%%
%%%%% How to convert this document to PDF %%%%%%%%
%%%%%%%%%%%%%%%%%%%%%%%%%%%%%%%%%%%%%%%%%%%%%%%%%%

% If your figures are stored as PostScript files, you can use the 
% following commands to generate a PDF file of your proposal:

%% latex file.tex
%% dvips file.dvi
%% ps2pdf file.ps file.pdf 


% If your figures are PDF images or bitmap pictures in PNG, JPG, or GIF format,
% you can use the pdflatex command to generate a PDF file from this template
% (note, however, that the pdflatex command does not handle PostScript files):

% pdflatex file.tex


% WARNINGS: 
%           1. You must make sure that PDF output generated from this
%              template is complete both when displayed with a viewer 
%              (acroread, for example) and when printed on paper.
%              LaTeX installations vary greatly and therefore it might 
%              not be possible to get all proposals to come out 
%              correctly with a single text page layout. 
%              In some cases you will have to adjust the 
%              \topmargin=-7mm command in the template to center the 
%              text vertically in the page.  
%           2. The scientific justification, figures, tables, references,
%              and public outreach statement must all fit within the
%              4-page limit.
%           3. You are free to include colour images in your proposal 
%              justification. Proposals are distributed to ALMA Review Panels 
%              in electronic form. However, the scientific content of the 
%              images should still remain clear when displayed or printed
%              in black and white.

%%%%%%%%%%%%%%%%%%%%%%%%%%%%%%%%%%%%%%%%%%%%%%
%%%%% Default format: 12pt single column %%%%%
%%%%%%%%%%%%%%%%%%%%%%%%%%%%%%%%%%%%%%%%%%%%%%

\documentclass[12pt,a4paper]{article}

\usepackage{graphics,graphicx}

%%%%%%%%%%%%%%%%%%%%%%%%%%%%
%%%%%% Page dimensions %%%%%
%%%%%%  DO NOT CHANGE  %%%%%
%%%%%%%%%%%%%%%%%%%%%%%%%%%%

\textheight=247mm
\textwidth=180mm
\topmargin=-7mm
\oddsidemargin=-10mm
\evensidemargin=-10mm
\parindent 10pt

%%%%%%%%%%%%%%%%%%%%%%%%%%%%%
%%%%% Start of document %%%%% 
%%%%%%%%%%%%%%%%%%%%%%%%%%%%%

\begin{document}
\pagestyle{plain}
\pagenumbering{arabic}
 
%%%%%%%%%%%%%%%%%%%%%%%%%%%%%
%%%%% Title of proposal %%%%%
%%%%%%%%%%%%%%%%%%%%%%%%%%%%%

\begin{center}
{\LARGE{\bf
%%
%% ENTER TITLE OF PROPOSAL BELOW THIS LINE
{Instabilities and origin of SiO in jet-driven molecular outflow}
%%
%%
}}
\end{center}
\bigskip

%% Principal Investigator (PI) initial(s) and family name %%
\centerline{\bf PI: 
%% ENTER NAME OF PI BELOW THIS LINE
{Thomas Douglas}}

\bigskip

% Type a concise abstract of your proposal here (optional).

\section{Abstract}
The onset of star formation is marked by the ejection of highly
collimated molecular outflows. These early jet-like outflows
are traced using standard outflow tracers like 
SiO and CO. We aim to study 
origin of SiO molecules in a classical jet-driven molecular outflow,
HH 212. In particular, we
propose high sensitive observation of regions close to symmetrical internal knots seen in
HH 212 with SiO (8-7) using the ALMA band 7.
Our recent numerical magneto-hydrodynamic model of 
jet propagation in conjuction molecular chemistry and cooling 
predicts high velocity narrow band emission 
from SiO molecules in the unshocked regions
within the primary jet beam. Observational evidences of the stated
prediction will confirm the formation of SiO molecules 
within the primary jet.
  
 


%%%%%%%%%%%%%%%%%%%%%%%%%%%%%%%%%%%%%%%%%
%%%%% Body of science justification %%%%%
%%%%%%%%%%%%%%%%%%%%%%%%%%%%%%%%%%%%%%%%%

%% ENTER TEXT, FIGURES AND TABLES BELOW

\section{Scientific Background}

Enter the scientific justification here, together with any figures and tables that you judge necessary.
 
%-----------------------------Figure Start---------------------------
\begin{figure}[tbh]
% The 'scale' parameter below allows you to scale the figure so that it fits within the page. In this case the figure was scaled to 20% of its original size.
%\includegraphics[scale=0.2]{CO_velfield.png}
%\caption{\em{The CO(1-0) velocity field of NGC\,3256, with contours 
%of the total line emission map overlaid (ALMA Science Verification Data).
%}}
\end{figure}
%-----------------------------Figure End------------------------------

% %-----------------------------Table Start-----------------------------
% \begin{table}[tbh]
% \begin{center}
% \caption[]{\em{Here we show the continuum sensitivity required per band.}}
% \begin{tabular}{cc}
% \hline \noalign {\smallskip}
% Frequency (GHz) & Sensitivity (mJy) \\
% \hline \noalign {\smallskip}
% 100 & 0.01 \\
% 300 & 0.10 \\
% %\hline \noalign {\smallskip}
% \end{tabular}
% \end{center}
% \end{table}
% %-----------------------------Table End ------------------------------



\section{Scientific Objectives}
\begin{enumerate}
\item \textbf{\large{To determine the survival of molecules in jets.}}



\item \textbf{\large{To observe the thermal instabilities in young
    molecular jets.}}


\item \textbf{\large{To provide constraints on jet-driven outflow models.}}
\end{enumerate}


\section{Technical Justification}

% Please describe the observations to be made and their specific
% purpose, with a clear explanation of the need for, and 
% appropriateness of, ALMA Cycle 1 data.  

%%%%%%%%%%%%%%%%%%%%%%%%%%%%%
%% Potential for Publicity %%
%%%%%%%%%%%%%%%%%%%%%%%%%%%%%

\section{Potential for Publicity}

% Here, include a brief statement on the potential of your proposal
% to generate publicity based on the scientific results to be obtained.


%%%%%%%%%%%%%%%%%%%%%%%%
%% References section: %
%%%%%%%%%%%%%%%%%%%%%%%%

\section{References}

% List references here

\noindent [1] Author1 et al. year, journal, vol, page

\noindent [2] Author2 et al. year, journal, vol, page


%%%%%%%%%%%%%%%%%%%%%%%%%%%
%%%%% End of document %%%%%
%%%%%%%%%%%%%%%%%%%%%%%%%%%

\end{document}

