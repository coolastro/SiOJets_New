\documentclass[8pt,xcolor=dvipsnames]{beamer}
\usepackage[latin1]{inputenc}

\usetheme[height=7mm]{Berlin}
\usecolortheme[named=Brown]{structure} 


\usepackage{time}       % date and time
\usepackage{graphicx}
\usepackage[T1]{fontenc}    % european characters
%\usepackage{courier}
\usepackage{animate}
\usepackage{multirow}
\usepackage{natbib}
\usepackage{amssymb,amsmath}  % use mathematical symbols
\usepackage{bookman}                  % use palatino as the default font
\setbeamercovered{transparent}
%\newcommand{\figpath}{/home/phybva/SiOJets_New/TALK/NEWFIGS}
\newcommand{\figpath}{./NEWFIGS/}
\newcommand{\jlfigpath}{/Users/bhargavvaidya/THESIS/chapter3/figures/}
\newcommand{\spic}[2]{\includegraphics[scale=#1]{#2}}
\newcommand{\myref}[1]{{\small{\color{red}{(#1)}}}}
\newcommand{\movauto}[5]{\animategraphics[width=#1,height=#2,autoplay]{#3}{#4}{0000}{#5}}
\newcommand{\movloop}[5]{\animategraphics[width=#1,height=#2,loop]{#3}{#4}{0000}{#5}}

\title[YSO Outflows]{Outflows from Young Stellar Objects}
\subtitle{MHD, Radiation \& Chemistry}
\author[Bhargav Vaidya]{\textcolor{blue}{Bhargav Vaidya}\inst{1}}
\institute[Uni. Leeds]{\inst{1}School of Physics and Astronomy, University of Leeds, Leeds.}
\date[2014]{{\large{Astro Seminar, TIFR}}\\ {\small{January 13, 2014}}}
%\titlegraphic{\vskip30pt \hskip215pt\includegraphics[width=0.3\textwidth]{\figpath/Leeds-logo.jpg}

\begin{document}

\begin{frame}
\vspace*{0.5cm}\titlepage  
  \vspace*{0.1cm}\textcolor{red}{{\textit{Collaborators:}}}\\
  \tiny{Paola Caselli (Uni. Leeds), Thomas Douglas (Uni. Leeds), Oliver Porth (Uni. Leeds), \\
  Christian Fendt (MPIA), Henrik Beuther (MPIA), Somayeah Sheiknezami (MPIA),\\
  Ciraco Goddi (JIVE),  Andrea Mignone (Uni. Torino).\\}
%\titlepage
\end{frame}

\begin{frame}
\frametitle{Outline}
\tableofcontents
\end{frame}

\section{Introduction}
\begin{frame}[plain]
\frametitle{Typical HII region}
\begin{center}
\spic{0.67}{\figpath/HII-region.jpg}
\end{center}
\end{frame}

\begin{frame}[plain]
\frametitle{Star formation : Dark clouds}
Filamentary, dense Infra-red dark clouds are birth place of stars
\myref{e.g.,Rathborne 2008}
\spic{0.15}{\figpath/g11_k.png}
\spic{0.3}{\figpath/g11_a_th.png}\\
\begin{center}
\spic{0.35}{\figpath/g11_850_th.png}
\end{center}
\end{frame}

\begin{frame}
\frametitle{Star formation : Feedback}
\begin{columns}
\begin{column}{0.45\textwidth}
\begin{block}{Outflows}
\begin{itemize}
\item An ubiquitous feature in star forming regions. First signpost of formation.
\item Driven magnetically from the underlying accretion disk. 
\item Entrain molecules along the flow : Molecular Outflow.
\item Stellar winds : Trigger star formation, Pollute ISM with new
  metals. 
\item 3 stages : Launching < 100 AU, Propogation few 100s AU to 0.1
  pc, Interaction > 0.1 pc.   
\end{itemize}
\end{block}
\end{column}
\begin{column}{0.55\textwidth}
M20 (Trifid Nebula) \spic{0.35}{\figpath/trigg_sf.jpg}\\
Jets from YSOs \spic{0.14}{\figpath/jets_yso.JPG}
\end{column}
\end{columns}
\end{frame}


%\begin{frame}{Present challenges}
%\end{frame}


\section{Motivation}
\begin{frame}{Outflow Evolutionary Picture \myref{Beuther \& Sheperd
      2005, Vaidya 2011}}
\begin{center}
\spic{0.1}{/Users/bhargavvaidya/THESIS/Intro_Figs/evol_outflow_n.pdf}
\end{center}
\end{frame}

\begin{frame}{Global Outflow Picture}
\begin{columns}
\begin{column}{1.0\textwidth}{\myref{Vaidya 2009}}
\spic{0.33}{/Users/bhargavvaidya/THESIS/chapter2/figures/f5_col.pdf}
\end{column}
\end{columns}
\end{frame}

\begin{frame}{Chemistry in outflows}
\begin{block}{Young Class 0 Outflows}
\begin{itemize}
\item Largely studied in form of molecular outflows using sub-mm
  telescopes.
\item Bulk gas motion is traced by CO while SiO and H$_{2}$ (Infrared) traces the shocked regions. 
\end{itemize}
\end{block}

\begin{block}{Class I jets}
\begin{itemize}
\item They are traced mainly using optical telescopes.
\item Signatures of forbidden emission lines from various atomic
  species.
\item Lack or neglible molecular emission. 
\end{itemize}
\end{block}
\end{frame}

\begin{frame}{Molecular bullets and EHV emission}
\begin{columns}

\begin{column}{0.4\textwidth}
\begin{center}
EHV Emission
\myref{Tafalla 2011}\\
\spic{0.33}{\figpath/EHVJets.png}
\end{center}
\end{column}
\hspace{0.05\textwidth}
\begin{column}{0.5\textwidth}
\begin{center}
PV Diagram \myref{Santiago-Garcia 2009}\\
\spic{0.25}{\figpath/EHVJets_PV.png}\\
1D Models fail \myref{Tafalla 2013} \\
\spic{0.35}{\figpath/EHVJets_Spec.png}
\end{center}
\end{column}
\end{columns}
\end{frame}

\begin{frame}{Goals}
\begin{block}{Outflow Evolution}
\begin{itemize}
\item <1-> What is the physical motivation of widening of outflow with age and central mass.?
\item <2-> Does the central massive star and hot inner disk play a role?  
\item <3->How do magnetic fields fit into the picture of outflows dynamics?
\end{itemize}
\end{block}

\begin{block}{EHV Emission}
\begin{itemize}
\item <4->What causes the EHV emission in young outflows?
\item <5->Is 2D the answer to the failure of 1D models to predict spectra?
\item <6->What happens if we extend the model to 3D? 
\end{itemize}
\end{block}
\end{frame}


\section{Methods : Numerical Simulations}
\begin{frame}{Numerical code}
\begin{columns}
\begin{column}{0.5\textwidth}
\begin{block}{PLUTO}
\begin{itemize}
\item A modular code for computational astrophysics.
\item Solve systems of conservation laws using the finite volume or
  finite difference approach based on Godunov-type schemes.
\item The static grid version of PLUTO is entirely written in the C programming language
\item AMR module uses the Chombo library. 
\end{itemize}
\end{block}
\end{column}
\begin{column}{0.5\textwidth}
\spic{0.26}{\figpath/Pluto_code_cover.png}
\end{column}
\end{columns}
\end{frame}




\begin{frame}{Launching Model}
\begin{columns}
\begin{column}{0.45\textwidth}{\textit{\color{red}{Cartoon Model}}}
\spic{0.2}{\jlfigpath/f1.pdf}
\end{column}
\hspace{0.05\textwidth}
\begin{column}{0.45\textwidth}{\textit{\color{red}{Simulation Box}}}
\spic{0.3}{\jlfigpath/f2.pdf}
\end{column}

\end{columns}
\end{frame}


\section{Outflow Dynamics : Launching}
\begin{frame}{Radiation force}
\begin{block}{Castor Abbott \& Klein Model}
\begin{itemize}
\item \textit{Line Driving force:} Main driving mechanism from evolved
  massive stars.
\item \textit{Physics:} Momentum transfer from photon to the gas. 
  Optical depth depends on local flow variables - Sobolev Approximation.
\item \textit{Formulae:} $f_{\rm line} = f_{\rm cont} \mathcal{M}(t)$
where $t$ is a function of $Q_0$ and $\alpha$.
\end{itemize}
\end{block}
\begin{center}
\begin{tabular}{l l}
\hline
Quantity & Value\\
\hline\hline
Stellar Mass & 30 M$_{\odot}$\\
Plasma $\beta$ & 5.0\\
R$_{\rm in}$ & 1.0 AU \\
Eddington $\gamma$ & 0.236 \\
Q$_0$ & 1400.0 \\
$\alpha$ & 0.55 \\
$\rho_0$ [g cm$^{-3}$] & $5.0\times10^{-14}$\\
\hline
\end{tabular}
\end{center}

\end{frame}


\begin{frame}{MHD Acceleration}
\begin{columns}
\begin{column}{0.4\textwidth}
\begin{block}{Impact of Stellar radiation force.}
\begin{itemize}
\item Initial MHD flow achieves a steady state.
\item Jet perturbed after MHD steady state : Addition of radiation force from star or disk. 
\item The flow ultimately achieves a steady state \textbf{BUT} with increase in
  jet velocity.
\item Intensity of the force scales with the jet launching radius. 
\end{itemize}
\end{block}
\end{column}
\begin{column}{0.6\textwidth}
\spic{0.23}{\jlfigpath/f6_col.pdf}
\end{column}
\end{columns}
\end{frame}

\begin{frame}{Collimation and Radiation \myref{Vaidya et. al. 2011}}
\begin{columns}
\begin{column}{0.4\textwidth}
\begin{block}{Impact of Stellar radiation force.}
\begin{itemize}
\item Star's mass and luminosity - \myref{Hosokawa 2008}
\item Collimation : Opening angles at critical points in the steady
  flow.
\item Without radiation -- Alfven point : Dotted line,
  Magnetosonic-fast : Solid line.
\item With radiation -- Alfven point : Stars, Magnetosonic-fast : dots.
\item Radiation force from Central Star -- \textbf{WIDENS THE FLOW}
\end{itemize}
\end{block}
\end{column}
\begin{column}{0.6\textwidth}
\spic{0.23}{\jlfigpath/f11.pdf}
\end{column}
\end{columns}
\end{frame}

\begin{frame}{Force Parameters and its impact}
\begin{tabular}{l l}
Magnetic field & Density and $\alpha$ \\
\spic{0.18}{\jlfigpath/f9.pdf} & \spic{0.18}{\jlfigpath/f10.pdf}
\end{tabular}
\end{frame}

\begin{frame}{Fixing mass flux?}
\begin{tabular}{l l}
Floating mass flux & Direct Influence \\
\spic{0.18}{\jlfigpath/f12.pdf} & \spic{0.18}{\jlfigpath/f13_col.pdf}
\end{tabular}
\end{frame}

\begin{frame}{Resistive effects}
Diffusive Disk Jet Launching \myref{Sheikhnezami et. al 2012} 
\begin{center}
\spic{0.5}{\figpath/Res_diskjet.png}
\end{center}
\end{frame}

\begin{frame}{A case of Orion Source I}
\begin{columns}
\begin{column}{0.35\textwidth}{Orion Source I - Masers}
\spic{0.06}{/Users/bhargavvaidya/MyProject/work/Orion_BN_KL/MNRAS/Revised/f1.pdf}
\end{column}
\hspace{0.05\textwidth}
\begin{column}{0.55\textwidth}{Simulation Model}
\spic{0.07}{/Users/bhargavvaidya/MyProject/work/Orion_BN_KL/MNRAS/Revised/f2.pdf}
\end{column}
\end{columns}
\end{frame}

\begin{frame}{A case of Orion Source I}
\begin{block}{Simulation Results \myref{Vaidya 2013}}
\begin{itemize}
\item SiO Masers :  n (H$_{2}$) $\sim 10^{9} \rm cm ^{-3}$ and T $\sim$ 2000 K \myref{Goddi et. al. 2009} 
\item \textit{Dark band} - R $<$ 14 AU. T and $\rho$ optimum to
  produce SiO masers within 15-60 AU : Consistent with Observations 
\item 3D Velocities lie within observed range and show signatures of
  rotation \myref{Matthews 2010}
\end{itemize}
\end{block}
\begin{columns}
%\hspace{0.1\textwidth}
\begin{column}{1.0\textwidth}
\spic{0.27}{/Users/bhargavvaidya/MyProject/work/Orion_BN_KL/MNRAS/Revised/f3.pdf}
\end{column}
\end{columns}
\end{frame}

\section{Outflow dynamics : Propagation}

\begin{frame}{Chemistry in Jets}
\begin{block}{Initial Setup}
\begin{itemize}
\item Young jets -- largely atomic \myref{Dionatos 2009}, Ionization
  fraction around 1\% \myref{Bacciotti 1995}, Molecules survive in jet
  \myref{Glassgold 1991}
\item Axisymmetric dense jet ($\eta$ > 1) enters into a fully molecular ambient
  medium with density varying in height - $\rho_{\rm amp} \sim z^{-2}$
\item Simplified Hydrogen chemistry : Tracks fraction of H$_2$, HI and
  HII. 
\end{itemize}
\end{block}
\begin{columns}
\hspace{-0.1\textwidth}
\begin{column}{0.95\textwidth}
\tiny{
\begin{tabular}{l l l l}
\hline
No. & Reaction & Rate Coefficient ($\rm {cm}^{3} s^{-1}$) &
Reference~\footnote{REFERENCES -- (1) \myref{Cen 1992 [Eq. 26a]};
  (2) \myref{Woodall 2007 [UMIST Database]} (3)
  \myref{Galli 1998 [Eq. H17]}; (4) \myref{Abel 1997
  [Tab. 3 Eq. 13]}; (5) \myref{Hollenbach 1979 [Eq. 3.8]}}\\
\hline
1. & H + e$^{-}$ $\rightarrow$ H$^{+}$ + 2e$^{-}$ & $k_1$ = $5.85
\times 10^{-11}$ $T^{0.5}$ \rm{exp}(-157,809.1/T)/(1.0 + $T_{5}^{0.5}$) & 1\\
2. & H$^{+}$ + e$^{-}$ $\rightarrow$ H + h$\nu$ & $k_2$ =
$3.5\times10^{-12} (T/300.0)^{-0.8}$ & 2\\
3. & H$_{2}$ + e$^{-}$ $\rightarrow$ 2H + e$^{-}$ & $k_3$ =
$4.4\times10^{-10} T^{0.35} \rm{exp}(-102,000.0/T)$ & 3\\
4. & H$_{2}$ + H $\rightarrow$ 3H & $k_4$ = $1.067\times10^{-10}
T_{\rm eV}^{2.012}(\rm{exp}(4.463/T_{\rm eV})^{-1}((1. + 0.2472 T_{\rm eV})^{3.512})^{-1} $& 4\\
5. &H$_{2}$ + H$_{2}$ $\rightarrow$ H$_{2}$ + 2H & $k_5$ = $1.0\times 10^{-8} \rm{exp}(-84,100/T)$ & 2\\
6. & H + H $\overset{\rm dust}\longrightarrow$ H$_{2}$ & $k_6 =
3.0\times10^{-17}\sqrt{T_{2}}(1.0 + 0.4\sqrt{T_{2} + 0.15} + 0.2T_{2} + 0.8T_{2}^{2})$ & 5 \\
\hline
\end{tabular}
}
\end{column}
\end{columns}
\end{frame}

\begin{frame}{Cooling in Jets}
\begin{columns}
\begin{column}{0.3\textwidth}
\vspace{-3cm}
\begin{block}{Cooling is important}
\begin{itemize}
\item t$_{\rm cool} \sim n_0/\Lambda(T) <$ t$_{\rm dyn}$
  \myref{e.g, Blondin, 1990}\\
\item Cooling rate for a molecular medium with n$_0$ = 10$^{5}$ cm$^{-3}$
\end{itemize}
\end{block}
\end{column}
\begin{column}{0.6\textwidth}
\spic{0.3}{\figpath/pfig1.pdf}
\end{column}
\end{columns}
\end{frame}

\begin{frame}{Cooling in Jets}
\begin{block}{Dynamical Effects}
\centering
\begin{itemize}
\item Thinner jets with less pronounced coccoon
\item Enhanced density in the internal knots with instable features.
\item Thermal Instability in bow-shock of the jet. 
\end{itemize}
\end{block}
\centering
\spic{0.23}{\figpath/pfig4.pdf}

\end{frame}

\begin{frame}{Molecular Interplay}
\begin{columns}
\begin{column}{0.4\textwidth}
\spic{0.2}{\figpath/pfig6.pdf}
\end{column}
\hfill
\begin{column}{0.6\textwidth}
\spic{0.2}{\figpath/pfig7.pdf}
\end{column}
\end{columns}
\end{frame}



\begin{frame}{SiO Abundance and Jet Velocity}
\begin{columns}
\begin{column}{0.5\textwidth}
\begin{block}{SiO in outflows}
\begin{itemize}
\item SiO production by grain-grain collision \myref{Caselli 1997},
  mantle evaporation \myref{Schilke 1997},
  gas-grain interaction (sputtering) \myref{Gusdorf 2008} 
\item Empirical dependence of n(SiO)/n(H$_2$) on flow velocity.
\item SiO Abundance - 10$^{-12}$ in quiescent dark clouds
  \myref{Ziurys 1989}
\item SiO Abundance - 10$^{-7}$ to 10$^{-6}$ in outflows with flow
  speeds > 60 km s$^{-1}$ e.g. L1448 \myref{Martin-Pintado 1992,
    Dutrey 1997}
\end{itemize}
\end{block}
\end{column}
\begin{column}{0.5\textwidth}
\spic{0.2}{\figpath/pfig3.pdf}
\end{column}
\end{columns}
\end{frame}

\begin{frame}{Spectra and PV diagrams : SiO (2-1)}
\begin{columns}
\begin{column}{0.48\textwidth}
\spic{0.17}{\figpath/pfig8.pdf}\\
\hspace{0.2cm} $\phi = \pi/2$ (Plane of Sky)
\end{column}
\hfill
\begin{column}{0.48\textwidth}
\spic{0.17}{\figpath/pfig9.pdf}\\
\hspace{0.2cm}$\phi = \pi/4$ 
\end{column}
\end{columns}
\end{frame}

\begin{frame}{Multi-Line survey : Emission I}
\small{
\begin{itemize}
\item Molecular Jet with $\eta$ = 3.0\\
\item Line Intensities with Top Hat Profile. \\
\item Jet is in plane of sky and convolved with a Gaussian beam of 2.5''
\end{itemize}
}
\begin{tabular}{ccc}
\textbf{J = 2-$>$1} & \textbf{J = 5-$>$ 4} & \textbf{J = 8-$>$ 7}\\
\movauto{3.5cm}{5.5cm}{3}{Img21/Img21_}{0010}
&
\movauto{3.5cm}{5.5cm}{3}{Img54/Img54_}{0010}
&
\movauto{3.5cm}{5.5cm}{3}{Img87/Img87_}{0010}

\end{tabular}

\end{frame}

\begin{frame}{Multi-Line survey : Emission II}
\begin{block}{Multi-line transitions : HH 211}
\begin{itemize}
\item Lower energy transitions - SiO J=1-0,2-1 show more turbulent
  features.\myref{Chandler 2001}
\item Higher energy transitions - SiO J=5-4,8-7 are more compact and
  trace the inner most jet \myref{Hirano 2006, Palau 2006, Nisini 2007}
\end{itemize}
\end{block}
\begin{columns}
\begin{column}{0.95\textwidth}
\spic{0.24}{/Users/bhargavvaidya/MyProject/work/Leeds_Uni/SiOJets_New/PAPER/PFIGS/imshkvcontomaps.png}
\end{column}
\end{columns}
\end{frame}

\begin{frame}{Case of L1448 \myref{Nisini 2007}}
\begin{columns}[t]
\begin{column}{0.25\textwidth}
\spic{0.1}{\figpath/Nisini2007_f1.pdf}\\
Contours SiO 2-1
\end{column}
%\hfill

\begin{column}{0.40\textwidth}
 \spic{0.25}{\figpath/Nisini2007_f4.pdf}\\
 Spectral Features
\end{column}
%\hfill

\begin{column}{0.25\textwidth}
\spic{0.25}{\figpath/Nisini2007_f6.pdf} \\
Kinematic Study (LVG) 
\end{column}
%\hfill

\end{columns}
\end{frame}

\begin{frame}{Multi-line survey : Line Ratios}

\begin{columns}[T]
\begin{column}{0.35\textwidth}
\begin{itemize}
\item EHV emission of 0.5 K.
\item Line ratios close to Unity. 
\item Multi-line emission show a distinct fall at high J$_{\rm up}$.
\end{itemize}
\end{column}
\hfill


\begin{column}{0.75\textwidth}
\spic{0.28}{\figpath/pfig14.pdf} 
\end{column}
\hfill


\end{columns}

\end{frame}

\begin{frame}{Predictions for ALMA}
\begin{block}{ALMA Cycle 2}
\textbf{Bands 3, 6, 7}: Assuming a source placed at 400 pc.[Orion - HH 212 \myref{Zinnecker,1998}] 
\end{block}
\begin{center}
\spic{0.21}{\figpath/pfig12.pdf}
\end{center}
\end{frame}

\begin{frame}{Focussing on a single knot}
\begin{tabular}{c c}
\spic{0.2}{\figpath/codella_hh212_f3.pdf} & \spic{0.2}{\figpath/knot_ang85.pdf} \\
HH 212 \myref{Codella 2007} &\\
\end{tabular}
\end{frame}

\begin{frame}[t]{Rotation or Wiggles?}
\begin{columns}
\begin{column}{0.3\textwidth}
\begin{block}{Work in Progress}
\begin{itemize}
\item Extention to 3D opens up slew of instabilities.
\item Self generated internal shocks create stationary knots.
\item Wiggles are evident at later stages of evolution. 
\end{itemize}
\end{block}
\end{column}
\begin{column}{0.7\textwidth}
  \movloop{8cm}{5cm}{61}{Img_3djet/contourjet1.}{0060}
\end{column}
\end{columns}
\end{frame}



\section{Summary}
\begin{frame}[plain]
\frametitle{Conclusions}
\begin{block}{Outflow Evolution}
\begin{itemize}
\item Radiative Line force from star plays a crucial role in widening
  the flow for stars with mass $>$ 30 M$_{\odot}$.
\item The Collimation degree increases with increase in mass
  (luminosity) and decrease in magnetic fields, density and line force
  parameters.
\item \textit{Radiation MHD Interplay} :  Magnetic fields initially drive the outflow and then radiative
  force takes over to compliment the dynamics.
\end{itemize}
\end{block}
\begin{block}{EHV Emission}
\begin{itemize}
\item Axisymmetric jet driven molecular outflows can very well
  reproduce features of EHV emission.
\item PV diagrams show a distinct zig-zag pattern specific of EHV
  emission from young outflows.
\item Higher energy transition lines traces compact knots in the flow
  while jet interaction regions are traced by low excitation lines.
\item \textit{Chemistry and Dyanmics} : Synthetic observations by
  combining chemistry with dynamical models provides a link between
  future observations and theory. 
\item 3D Modelling with enhanced chemical network is the way forward.
\end{itemize}
\end{block}
\end{frame}

\end{document}

